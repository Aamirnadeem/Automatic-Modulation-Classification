\documentclass[10pt,twocolumn]{witseiepaper}

\usepackage{KJN}

\ifpdf
\pdfinfo{
/Title (ELEN4012 - Feature Based Automatic Modulation Classification)
/Author (Jacques Visser and Anthony Farquharson)
}
\fi

\begin{document}

\title{ELEN4012 - Feature Based Automatic Modulation Classification}

\author{Jacques Visser and Anthony Farquharson
\thanks{School of Electrical \& Information Engineering, University of the
Witwatersrand, Private Bag 3, 2050, Johannesburg, South Africa}
}

% TODO Rewrite abstract once the rest of the contents are figured out.
\abstract{Automatic modulation classification involves identifying the modulation scheme used in a signal without the decision being guided by an operator. This report covers a preliminary investigation into the design and implementation of such a system. An overview of the relevant literature is presented and proposals are made regarding the details of the implementation and testing of such a system using and Ettus USRP.}

\keywords{modulation, classification, USRP, UHD}

\maketitle
\thispagestyle{empty}\pagestyle{empty}

% TODO How does one even write an introduction?
\section{INTRODUCTION}

\section{LITERATURE SURVEY}
Zhu and Nandi \cite{zhu2014automatic} identifies three major approaches to automatic modulation classification; Likelihood-based, distribution test-based and feature-based. These are briefly detailled below.
	\subsection{Likelihood Based Classification}
	\subsection{Distribution Test Based Classification}
	\subsection{Feature Based Classification}
	Feature based AMR has been shown to be non-ideal, but significantly less computationally intensive \cite{zhu2014automatic} than the aforementioned methods.

	There are again three major approaches to feature-based AMC. These make use features derived from either the signal spectrum, the wavelet transform of the signal or high-order statistical representations of the signal \cite{zhu2014automatic}. 
	
	The classification of analog modulation schemes using spectral features is well documented by Zhu and Nandi \cite{zhu2014automatic} as well as Azzouz and Nandi \cite{azzouz2013automatic}.
	

\section{EXISTING SOLUTIONS AND APPLICATIONS OF AMC}
	\subsection{Military}
		% jamming, listening in on communications
	\subsection{Civilian}
		% cognative radio, effective use of bandwith

\section{DESIGN PROCESS OVERVIEW}
	\subsection{Development Methodology}
		% Agile? More of an engineering-oriented method
	\subsection{Estimated Project Schedule}
		% Design, Implementation, Simulated Testing, Iteration, Practical Testing and Evaluation
	\subsection{Estimated Costs and Hardware Required}
		% Two USRP's and two ariels

\section{IMPLEMENTATION OVERVIEW}
	\subsection{Hardware}
		% USRP
		% Antenna?
		% Runs on a Linux pc, not win cos of UHD

	\subsection{Software}
		\subsubsection{Basic Software Structure}
			% fundamentally threaded
			% UHD on one thread
			% each of the feature extraction functions on their own thread
			% classifier on a final thread
		\subsubsection{Libraries and API's}
			% UHD API
			% FFT Library
			% SFML for gui and plotting
			% C++11 std
		\subsubsection{Build System}
			% Cmake because easy, also can be compiled on windows (probably UHD issue, though)

\section{PROPOSED TESTING PROCEDURE}
	\subsection{Simulated Testing}
		% Evaluate system with generated signals
		% See how well it fares with different number of samples
		% Signal to noise ratio
	\subsection{Practical Testing}
		% correlate real world results with simulation
		% Don't make a new program, just use GNURadio companion to make signals to classify

\section{CONCLUSION AND RECOMMENDATIONS}


\bibliographystyle{witseie}
\bibliography{prelim} 
\end{document}
