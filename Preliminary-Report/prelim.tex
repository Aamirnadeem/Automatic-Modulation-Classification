\documentclass[10pt,twocolumn]{witseiepaper}

\usepackage{KJN}

\ifpdf
\pdfinfo{
/Title (ELEN4012 - Feature Based Automatic Modulation Classification)
/Author (Jacques Visser and Anthony Farquharson)
}
\fi

\begin{document}

\title{ELEN4012 - Feature Based Automatic Modulation Classification}

\author{Jacques Visser and Anthony Farquharson
\thanks{School of Electrical \& Information Engineering, University of the
Witwatersrand, Private Bag 3, 2050, Johannesburg, South Africa}
}

% TODO Rewrite abstract once the rest of the contents are figured out.
\abstract{Automatic modulation classification involves identifying the modulation scheme used in a signal without the decision being guided by an operator. This report covers a preliminary investigation into the design and implementation of such a system. An overview of the relevant literature is presented and proposals are made regarding the details of the implementation and testing of such a system using and Ettus USRP.}

\keywords{modulation, classification, USRP, UHD}

\maketitle
\thispagestyle{empty}\pagestyle{empty}

% TODO How does one even write an introduction?
\section{INTRODUCTION}

\section{LITERATURE SURVEY}
Zhu and Nandi \cite{zhu2014automatic} identifies three major approaches to automatic modulation classification; likelihood-based, distribution-test-based and feature-based. These are briefly detailled below.
	\subsection{Likelihood Based Classification}
	\subsection{Distribution Test Based Classification}
	\subsection{Feature Based Classification}
	Feature based AMR has been shown to be non-ideal, but significantly less computationally intensive \cite{zhu2014automatic} than the aforementioned methods.

	There are again three major approaches to feature-based AMC. These make use features derived from either the signal spectrum, the wavelet transform of the signal or high-order statistical representations of the signal \cite{zhu2014automatic}. 
	
	The classification of analog modulation schemes using spectral features is well documented by Zhu and Nandi \cite{zhu2014automatic} as well as Azzouz and Nandi \cite{azzouz2013automatic}.
	

\section{EXISTING SOLUTIONS AND APPLICATIONS OF AMC}
	\subsection{Military}
		% jamming, listening in on communications
	\subsection{Civilian}
		% cognative radio, effective use of bandwith

\section{DESIGN PROCESS OVERVIEW}
	\subsection{Development Methodology}
		% Engineering oriented method
		% make use of Trello
		% Add screenshots to appendix
	\subsection{Estimated Project Schedule}
		% Design, Implementation, Simulated Testing, Iteration, Practical Testing and Evaluation
		% Overview and Trello screenshots
	\subsection{Estimated Costs and Hardware Required}
		% Two USRP's and two antennas
		The practical implementation of this project will require at least two USRP devices. The first of which will be used for receiving radio signals, the modulation of which is to be classified. The second will be used to generate modulated signals in order to practically test the operation of the system. Seeing as Wits has these available, no charges would be incurred.

		Furthermore, at least two computers running Linux will be required. The team-members' labptops will be used for this purpose.

\section{IMPLEMENTATION OVERVIEW}
	\subsection{Hardware}
		% USRP
		% Antenna?
		% Runs on a Linux pc, not win cos of UHD
	As mentioned before, two USRP's and two computers are neccesary. To avoid significant effort in compiling and installation of the USRP Hardware Driver (UHD) both computers will run Linux natively.

	\subsection{Software}
		\subsubsection{Basic Software Structure}
			Due to the computationally intensive and time-dependent nature of the system, the software would have to be threaded, with each major component running on it's own thread.

			The software would consist of various components centered around a main control loop. The UHD driver, run in it's own thread, passes the signal to the central control stucture, which filters the signal and passes it to various feature extraction functions, all run in parralel. These functions deliver the features that have been extracted form the signal to a classifier which finally determines the modulation scheme used. This process is represented as a flow diagram in Figure \ref{fig:sw_overview}
			% fundamentally threaded
			% UHD on one thread
			% each of the feature extraction functions on their own thread
			% classifier on a final thread
		\subsubsection{Libraries and API's}
			Due to the complicated nature of the software implementation, all of the components will not be locally developed. Rather, various API's and libraries will be used.

			Firstly, the Ettus USRP Hardware Driver's (UHD) C++ API will be used for communication with the USRP. This is a free \& open source piece of software released under the GPLv3 license \cite{uhd_license}, and is thus free to be used for a project such as this.

			In order to spectrally analyse the incoming signal the KissFFT library is to be used. KissFFT is developed by Mark Borgerding and is available under a Revised BSD license \cite{kissfft_license}

			For displaying graphical information SFML (Simple and Fast Multimedia Library) will be used. This is distributed under the zlib/libpng License and may thus be freely used in a project such as this \cite{sfml_license}.

			For threading and other miscelaneous functions the C++ 11 Standard library will be used. GNU libstdc++ which is used with GCC is distributed under the GPLv3 license.

		\subsubsection{Build System}
			Due to the open-source nature and wide availability of the libraries and API's used in this project the software produced will be able to run on various platforms. To fully support this the CMake build-system-generator will be used so that this project may easily be compiled on various platforms.

			CMake is distributed under the BSD 3-Clause license \cite{cmake_license} and it can thus be distributed with and used in this project.
			% Cmake because easy, also can be compiled on windows (probably UHD issue, though)

\section{PROPOSED TESTING PROCEDURE}
	\subsection{Simulated Testing}
		% Evaluate system with generated signals
		% See how well it fares with different number of samples
		% Signal to noise ratio
	\subsection{Practical Testing}
		% correlate real world results with simulation
		% Don't make a new program, just use GNURadio companion to make signals to classify

\section{CONCLUSION AND RECOMMENDATIONS}


\bibliographystyle{witseie}
\bibliography{prelim} 
\end{document}
